Perhaps the most common of all software engineering activities is the modifications of software. Unfortunately, software modification, i.e., software maintenance, often leaves behind software that is difficult to understand for those other than its author. In this context, software restructuring is a field that seeks to reverse these effects on software. 

More specifically, software restructuring is the modification of software to make the software easier to understand and to change, or less susceptible to error when future changes are made (ref). In other words, it is the process of re-organizing the logical structure of existing software system to improve specific attributes~\cite{kang1999}. Some examples of software restructuring are improving coding style, editing documentation, transforming program components (moving class, creating class, etc). The central idea of restructuring is the action of transformation. According to~\cite{Eloff2002} a transformation can be defined formally as a function that receives a program, P, as input and produces a new program, P'. Thus, P' is said to be functionally equivalent to P $\Leftrightarrow$ P' exhibits identical behavior  to P for all defined inputs of P. Finally, T is called a meaning preserving transformation if P' $\equiv$ P.

According to Griswold's experiments programmers tends to not only commit syntactic errors and behave inconsistently, they usually ignore the global impact of the changes they make 

Manually restructuring software may have undesirable, and often unforeseen results that can affect the be- haviour of a system. Griswold’s experiments found that programmers not only commit syntactic errors and behave inconsistently, but they also ignore the global impact of the changes they make [Griswold 1991]. Fur- thermore, manual techniques demand the maintainer to guarantee the preservation of the system’s behaviour.
