

\begin{abstract}
Today, almost every company runs systems that have been implemented a long time ago. These systems usually are still under adaptation and maintenance to address current needs. Very often, adapting legacy software systems to new requirements needs to make use of new technological advances. Furthermore, legacy software systems mainly consist of two kinds of artifacts: source code and databases. Usually, the maintenance of those artifacts is carried out through restructuring processes in isolated manners. Nevertheless, for a more effective maintenance of the whole system both should be analyzed and evolved jointly. Therefore, the lifespan of the legacy software systems are expected to improve. This paper proposes an approach to assist the modernization of both source code and database legacy systems. For this purpose, our technique analyses SQL queries embedded in the legacy source code in order to restructure and re-organize the legacy system by using design patterns. In  order to validate our approach we have carried out an experiment throughout a real-life case study. The results were promising regarding the effort employed to modernize a legacy software system.
\end{abstract}
% IEEEtran.cls defaults to using nonbold math in the Abstract.
% This preserves the distinction between vectors and scalars. However,
% if the conference you are submitting to favors bold math in the abstract,
% then you can use LaTeX's standard command \boldmath at the very start
% of the abstract to achieve this. Many IEEE journals/conferences frown on
% math in the abstract anyway.

% no keywords




% For peer review papers, you can put extra information on the cover
% page as needed:
% \ifCLASSOPTIONpeerreview
% \begin{center} \bfseries EDICS Category: 3-BBND \end{center}
% \fi
%
% For peerreview papers, this IEEEtran command inserts a page break and
% creates the second title. It will be ignored for other modes.
\IEEEpeerreviewmaketitle

\section{Introduction}
	\input{tex/introduction}

	%This paper is structured as follows: in Section II, Crosscutting Frameworks are explained; in Section III, the Proposed Model and the Reuse Process are shown; in Section IV, a tool to support the process is used to reuse a persistence framework as an Example; in Section V, an empirical evaluation is presented; in Section VI, there are related works and in Section VII, there are the conclusions.
	
	
\section{Software Restructuring\label{software restructuring}}
	Perhaps the most common of all software engineering activities is the modifications of software. Unfortunately, software modification, i.e., software maintenance, often leaves behind software that is difficult to understand for those other than its author. In this context, software restructuring is a field that seeks to reverse these effects on software. 

More specifically, software restructuring is the modification of software to make the software easier to understand and to change, or less susceptible to error when future changes are made (ref). In other words, it is the process of re-organizing the logical structure of existing software system to improve specific attributes~\cite{kang1999}. Some examples of software restructuring are improving coding style, editing documentation, transforming program components (moving class, creating class, etc). The central idea of restructuring is the action of transformation. According to~\cite{Eloff2002} a transformation can be defined formally as a function that receives a program, P, as input and produces a new program, P'. Thus, P' is said to be functionally equivalent to P $\Leftrightarrow$ P' exhibits identical behavior  to P for all defined inputs of P. Finally, T is called a meaning preserving transformation if P' $\equiv$ P.

According to Griswold's experiments programmers tends to not only commit syntactic errors and behave inconsistently, they usually ignore the global impact of the changes they make 

Manually restructuring software may have undesirable, and often unforeseen results that can affect the be- haviour of a system. Griswold’s experiments found that programmers not only commit syntactic errors and behave inconsistently, but they also ignore the global impact of the changes they make [Griswold 1991]. Fur- thermore, manual techniques demand the maintainer to guarantee the preservation of the system’s behaviour.


\section{Model-Driven Restructuring\label{MDD}}
	According to Griswold's experiments programmers tends to not only commit syntactic errors an behave inconsistently, they usually ignore the global impact of the changes they make during the restructuring process. Moreover, he also argues that manual restructuring is an error-prone and expensive activity~\cite{grisswold}. Therefore, software engineers have applied Model-Driven Development (MDD) technologies to software restructuring to deal with those limitations and to automatize the software restructuring. MDD consists of the combination of generative programming, domain-specific languages and model transformations. It also aims to reduce the semantic gap between the program domain and the implementation, using high-level models that shield software developers from complexities of the underlying implementation platform~\cite{France:2007:MDC:1253532.1254709}.
		
%\section{Evaluation}
%	\input{tex/evaluation}
%	\subsection{Reuse Study Definition}
%		\input{tex/reuse_study_definition}
%	\subsection{Maintenance Study Definition}
%		\input{tex/maintenance_study_definition}
%	\subsection{Study Planning}
%		\input{tex/study_planning}
%	\subsection{Operation}
%		\input{tex/operation}
%	\subsection{Data Analysis and Interpretation}
%		\input{tex/data_analysis_and_interpretation}
%	\subsection{Hypotheses Testing}
%		\input{tex/hypotheses_testing}
%	\subsection{Threats to Validity}
%		\input{tex/threats_to_validity}
\section{Related Work}
		\input{tex/related_works}
\section{Conclusions}
		\input{tex/conclusion}

\section*{Acknowledgments}
	\input{tex/acknowledgment}


